\documentclass[CMPE]{KGCOEReport}
\usepackage{float}
\usepackage{adjustbox}
\graphicspath{ {./images/} }

\newcommand{\name}{Mohammed Fareed \\ Trent Wesley}
\newcommand{\exerciseNumber}{5}
\newcommand{\exerciseDescription}{MSP432 Timers, Interrupts, and Analog-to-Digital
Converter}
\newcommand{\dateDone}{September 27, 2023}
\newcommand{\dateSubmitted}{October 18, 2023}

\newcommand{\classCode}{CMPE 460}
\newcommand{\LabSectionNum}{1}
\newcommand{\LabInstructor}{Prof.\ Hussin Ketout}
\newcommand{\TAs}{Andrew Tevebaugh \\  Colin Vo}
\newcommand{\LectureSectionNum}{1}
\newcommand{\LectureInstructor}{Prof.\ Hussin Ketout}


\begin{document}
\maketitle

\section*{Abstract}

\section*{Design Methodology}

Interrupts are a useful tool for efficiently handling events by pausing the CPU's current task to handle another. Interrupts provide the benefit of assigning priority to certain events so that they can be executed in a proper order. In addition, interrupts avoid the need to constantly poll, which can interfere with the flow of a program. Interrupts are used extensively in this exercise by timers, switches, UARTs, and GPIO.\\

The MSP432 contains two Timer32 modules for timing operations. Each module contains a counter which can be configured as a 16-bit or 32-bit down counter. In this exercise, pressing switch 1 on the MSP432 microcontroller board activates Timer32-1 which toggles LED1 every 0.5 seconds. To accomplish this, switch 1 was configured to generate interrupts on the falling edge of a button click. The interrupt from a button click would go the an interrupt service routine for the switch's port. Within the ISR, the interrupt flag for switch 1 is checked to check if the interrupt came from it. If the interrupt was from switch 1, the state of Timer32-1 gets toggled and a boolean value tracking Timer32-1's state is updated.\\

While Timer32-1 is active, it generates an interrupt every 0.5 seconds. Every interrupt triggers an ISR which checks the state of LED1 with a boolean variable and toggles both the variable and the LED state.\\

\section*{Results and Analysis}

\section*{Questions}

\section*{Conclusion}

\newpage
\begin{figure}[H]
    \centering
    \begin{adjustbox}{center}
        % \includegraphics[width=1.35\textwidth]{signoff.pdf}
    \end{adjustbox}
\end{figure}

\end{document}
