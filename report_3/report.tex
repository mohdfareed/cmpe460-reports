\documentclass[CMPE]{KGCOEReport}
\graphicspath{ {./images/} }

\newcommand{\name}{Mohammed Fareed \\ Trent Wesley}
\newcommand{\exerciseNumber}{3}
\newcommand{\exerciseDescription}{Characterization of OPB745}
\newcommand{\dateDone}{September 13, 2023}
\newcommand{\dateSubmitted}{September 20, 2023}

\newcommand{\classCode}{CMPE 460}
\newcommand{\LabSectionNum}{1}
\newcommand{\LabInstructor}{Prof.\ Hussin Ketout}
\newcommand{\TAs}{Andrew Tevebaugh \\ Ben Hyman \\ Colin Vo \\ Marshal Tiechman}
\newcommand{\LectureSectionNum}{1}
\newcommand{\LectureInstructor}{Prof.\ Hussin Ketout}


\begin{document}
\maketitle

\section*{Abstract}

In this laboratory exercise, the OPB745 photo-transducer was characterized by its ability to accurately measure the distance between it and a reflective surface. A test environment made of PVC pipes was made to isolate the sensor from outside light and adjust the distance between the OPB745 photo-transducer and aluminum foil used as a reflective surface. The distance of the reflective surface was varied from 0cm to 50cm. It was found that the voltage drop started high (around 5V) at 0mm indicating that no light was detected. The voltage drop reached a minimum at 5mm indicating that the light intensity detected was strongest at this distance. The voltage drop increased as the distance increased until about 45mm. A function generator was attached to the input of the 7406 inverter and frequency was increased from 100Hz until the output no longer resembled a square wave. It was found that the output waveform stopped resembling a square wave at around 1.2kHz with the 10k$\Omega$ load resistor and around 850Hz with the 20k$\Omega$ resistor.

\section*{Design Methodology}

To set up this exercise, a test environment was built to isolate the OPB745 sensor from outside light. This environment also provided the ability to change the distance between the OPB745 sensor and a reflective surface of aluminum foil inside, with measurement of the distance between them. A diagram of the test environment from the CMPE-460 manual is shown in figure \ref{fig:env}.

\begin{figure}[h]
    \centering
    \includegraphics[width=0.75\textwidth]{test_env.png}
    \caption{Test Environment Diagram}
    \label{fig:env}
\end{figure}

The test environment was built out of two PVC pipes. One pipe had a reflective aluminum surface on one end and could be slid into the other pipe with a larger diameter. A metric ruler was added to the inside of the inner pipe for measurement. A slot made of cardboard and tape was added to the end of the larger pipe through which the OPB745 sensor was placed.

The first part of this exercise tested the ability of the OPB745 to accurately measure the distance between it and a reflective surface with a 10k$\Omega$ and a 20k$\Omega$ load resistor. The circuit for this part is shown in figure \ref{fig:circuit1}.This circuit contains an OPB745 which emits and measures light. The value of R$_\text{f}$ was calculated to ensure that the diode receives a proper current. The value of R$_\text{f}$ was calculated using the forward voltage and the continuous forward current of the diode in the OPB745, which were retrieved from the data sheet of the sensor. The following is the calculation of the resistance of R$_\text{f}$:

\[ \frac{5\text{V} - \text{V}_\text{F}}{\text{I}_\text{F}} = \frac{5\text{V} - 1.70\text{V}}{40 \text{mA}} = 82.5 \Omega \]

This circuit was tested with the load resistor equal to 10k$\Omega$ and 20k$\Omega$. The voltage at the V$_{\text{out}}$ node was measured for select distances from 0mm to 50mm and the current across the load resistor was calculated.

\begin{figure}[h]
    \centering
    \includegraphics[width=0.5\textwidth]{circuit_1.png}
    \caption{Part 1 Circuit}
    \label{fig:circuit1}
\end{figure}

The second part of this exercise tested how well the output would retain its square wave characteristics as the frequency of a waveform generator was increased. The circuit for this part is shown in figure \ref{fig:circuit2}.

\begin{figure}[h]
    \centering
    \includegraphics[width=0.75\textwidth]{circuit_2.png}
    \caption{Part 2 Circuit}
    \label{fig:circuit2}
\end{figure}

As shown in figure \ref{fig:circuit2}, a waveform generator is attached to the input of an inverter. This inverter controls the current that will flow through the LED. Light from the LED hits the phototransistor and causes current to flow through the phototransistor. If enough current flows through the phototransistor, voltage at the input of the Schmitt trigger (74LS14) will be low causing V$_{\text{out}}$ to be high. If not much light hits the phototransistor, minimal current will pass through it. This means that the voltage at the input of the Schmitt trigger will be high and V$_{\text{out}}$ will be low. The resistance of R$_\text{f}$ was calculated using the same method as in part 1, taking into account the voltage drop across the inverting buffer (7406). The value of the voltage drop of the inverting buffer used is its low-output voltage retrieved from the data sheet. The following is the calculation of the resistance of R$_\text{f}$:

\[ \frac{5\text{V} - \text{V}_\text{F} - \text{V}_\text{OL}}{\text{I}_\text{F}} = \frac{5\text{V} - 1.70\text{V} - 0.70\text{V}}{40 \text{mA}} = 65 \Omega \]

The waveform generator was initially set to 100Hz with a duty cycle of 50\%. The frequency was increased by increments of 100Hz until the output no longer retained its square wave characteristics. The frequency at which the output no longer resembled a square wave was recorded. This was repeated with the load resistor equal to 10k$\Omega$ and 20k$\Omega$.

\section*{Results and Analysis}

Part 1 of this exercise consisted of increasing the distance between the OPB745 and the reflective surface. 10k$\Omega$ and 20k$\Omega$ resistors were tested as the load resistor and voltage at V$_{\text{out}}$ was measured. Table \ref{table:1} shows the distances tested and the output voltage. The current across the load resistor was calculated using Ohm's law.

\begin{table}[ht]
    \centering
    \caption{Results of Part 1.}
    \begin{tabular}{|c|c|c|c|c|}
        \hline
        & \multicolumn{2}{|c|}{$R_{L1} = 10k\Omega$} & \multicolumn{2}{|c|}{$R_{L1} = 20k\Omega$} \\
        \hline
        Distance (mm) & $V_{out}$ (V) & $I_{RL}$ (mA) & $V_{out}$ (V) & $I_{RL}$ (mA) \\
        \hline
        0 & 4.98 & 0.002 & 4.96 & 0.002 \\
        \hline
        1 & 1.65 & 0.335 & 2.61 & 0.1195 \\
        \hline
        2 & 0.73 & 0.427 & 0.68 & 0.216 \\
        \hline
        3 & 0.67 & 0.433 & 0.53 & 0.2235 \\
        \hline
        4 & 0.63 & 0.437 & 0.17 & 0.2415 \\
        \hline
        5 & 0.62 & 0.438 & 0.17 & 0.2415 \\
        \hline
        6 & 0.64 & 0.436 & 0.19 & 0.2405 \\
        \hline
        7 & 0.67 & 0.433 & 0.55 & 0.2225 \\
        \hline
        8 & 0.69 & 0.431 & 0.64 & 0.218 \\
        \hline
        9 & 0.71 & 0.429 & 0.67 & 0.2165 \\
        \hline
        10 & 0.72 & 0.428 & 0.69 & 0.2155 \\
        \hline
        11 & 0.74 & 0.426 & 0.70 & 0.22 \\
        \hline
        12 & 0.75 & 0.425 & 0.72 & 0.214 \\
        \hline
        13 & 0.76 & 0.424 & 0.73 & 0.2135 \\
        \hline
        14 & 0.77 & 0.423 & 0.74 & 0.213 \\
        \hline
        15 & 0.79 & 0.421 & 0.75 & 0.2125 \\
        \hline
        20 & 0.87 & 0.413 & 0.80 & 0.21 \\
        \hline
        25 & 2.45 & 0.255 & 0.86 & 0.207 \\
        \hline
        30 & 3.28 & 0.172 & 1.98 & 0.151 \\
        \hline
        35 & 3.77 & 0.123 & 2.80 & 0.11 \\
        \hline
        40 & 3.96 & 0.104 & 2.90 & 0.11 \\
        \hline
        45 & 3.96 & 0.104 & 2.96 & 0.102 \\
        \hline
        50 & 3.65 & 0.135 & 2.60 & 0.12 \\
        \hline
    \end{tabular}
    \label{table:1}
\end{table}

\section*{Questions}

\emph{1. In the lab we do not use a 7406 inverter on the output; instead we use the 74LS14 with Schmitt trigger. Why do we need to do this, and what is the difference between the two?}

\emph{2. Why does the voltage start at 5V at 0mm and then drop quickly? Why does it eventually go back to 5V?}

At 0mm, the sensor is in contact with the reflective surface, and thus senses no light. At the point it is no longer in contact with the reflective surface, it gets the maximum amount of light, resulting in the drop of voltage. As the sensor is moved away from the reflective surface, the amount of light absorbed by the sensor starts decreasing, increasing the voltage gradually.

\emph{3. Why does the frequency change when going from a 10K load resistor to a 20K load resistor? Did you anticipate it increasing or decreasing and why?}

\section*{Conclusion}

\end{document}
